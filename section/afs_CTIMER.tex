\section{Timer funktionalitet (CTIMER)}\label{afs:CTIMER}
For at opnå præcise timingsmuligheder opsættes CTIMER'eren, som er en 32-bit timer. Denne del bliver dog i den integrerede for det meste brugt til at kunne holde øje med om systemet er i live, når der f.eks. ikke er tilsluttet en host. I den diskrete version bruges CTIMER til generering af PWM, hvilket er beskrevet i afsnit \ref{afs:PWM}. \\

Timeren sættes op vha. \verb|TIMER_Init()| hvori de krævede parametre opsættes og derefter startes: 

\begin{itemize}[noitemsep]
	\item Timerens counter resettes. 
	\item Timerens counter sættes til ikke at stoppe ved match.
	\item Matchværdien sættes til det halve af timer clock frekvensen. (Ender ud i ca 500ms)
	\item Der angives at et match skal styre et output. (Specifik pin er angivet i \verb|pin_mux.h|)
	\item Pin skal initialiseres. 
	\item Interrupt er ikke ønsket og deaktiveres.
\end{itemize}