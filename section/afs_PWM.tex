\section{Pulse Width Modulation (PWM)}\label{afs:PWM}
På den diskrete version er CHARGE og DISCHARGE MOSFET'erne er altid tændte, såfremt en fejlsituation ikke er tilstede. For at detektere en fejlsituation skal en reference sættes, således at den maksimale op- og afladestrøm er veldefineret. PWM bliver ikke brugt i den integrerede version. \\

Referencen for den maksimale afladestrøm sættes ved hjælp af et PWM signal på $R4$ i schematic. Et PWM signal er nødvendig for systemets modularitet, da dennes værdi i softwaren kan ændres, hvorimod en reference opnået gennem for eksempel en spændingsdeler er fastlagt. Referencen for afladestrømmen udregnes i afsnit \ref{sec:overcurrent_dsg} til $1.7\volt$. Ligeledes bliver opladestrømmens reference udregnet i \ref{sec:overcurrent_chg} til at være $300\milli\volt$. PWM signalerne til referencerne genereres på følgende måde. \\

Der tages udgangspunkt i PWM signalet til kontrol af afladefetten, dvs. 1,7V. Duty cyclen udregnes: 

\begin {equation}
D = \frac{1.7\volt}{3.3\volt} = 0.515
\label{eq:duty}
\end {equation}\\

Da CTIMER'eren bruges til generering af PWM er opsætningen i grove træk det samme som beskrevet i afsnit \ref{afs:CTIMER}, dog med enkelte undtagelser. Default CTIMER config sættes med driveres indbyggede funktion \verb|CTIMER_GetDefaultConfig|. Herefter initialiseres CTIMER'en, og PWM signalet's egenskaber sættes i \verb|CTIMER_GetPwmPeriodValue|. Her benyttes $20\kilo\hertz$ som frekvens da det er standard for CTIMER'en. Dutycyclen specificeres også her, men rundes op til 0.52 (52 i koden da der ikke kan sættes decimaler).\\ 

PWM signalet sikres korrekt ved måling på oscilloskop, og der ses på figur \ref{fig:pwm_scope} at signalet er inden for fejlmargin, og at det giver en spænding på ca. $1.7\volt$. Grunden til at signalet ikke ligger helt præcist på $1.7\volt$ kan være at forsyningen til microen ikke er helt præcist på $3.3\volt$ men lidt under. \\

\begin{figure}[h]
	\centering
	\includegraphics[width=15cm]{billeder/pwm_scope.png}
	\caption{Screenshot fra oscilloskop af PWM signalet}
	\label{fig:pwm_scope}
\end{figure}

Da sikkerheden ved overstrøm har en hysterese, skal en reset mulighed realiseres. Dette gøres ved at rette microcontrolleren og efterfølgende sætte dutycyklen kortvarigt til 100\percent\space og derved sætte en puls på $3.3\volt$ ind på reference benet. I tilfælde af for høj eller for lav temperatur, eller når cellernes kapacitet er for lav, sættes dutycyklen til 0\percent, for at slukke for MOSFET'erne.

