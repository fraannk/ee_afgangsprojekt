\section{Sammenligning} \label{afs:sammenligning}
For at overskueliggøre sammenligningen af de to versioner, er specifikationerne sat op side om side i afsnit \ref{sec:sammenligning_specifikation} og kostprisen for dem er sat op i tabeller i afsnit \ref{sec:sammenligning_kostpris}. Der er stræbet efter at få samme funktioner og specifikationer på begge versioner for at gøre sammenligningen så fair som muligt. 

\section{Specifikationer} \label{sec:sammenligning_specifikation}

\begin{table}[h!]
	\small
	\centering
	\begin{threeparttable}
		\begin{tabular}{ l l l l l l l }
			\toprule
			\multicolumn{1}{l}{\textbf{Specifikation}}          &
			\multicolumn{1}{l}{\textbf{Diskret}}           &
			\multicolumn{1}{l}{\textbf{Integreret}}  &
			\multicolumn{1}{l}{\textbf{Kommentar}}   \\ 
			\hline
			Opladestrøm $I_{in}$           &  $1\ampere$            & $1\ampere$            &    Diskret slår fra for tidligt pga. en fejl\\
			Afladestrøm $I_{out}$          &   $6\ampere$           & $6\ampere$            &    Diskret slår fra for tidligt pga. en fejl\\
			Balanceringsstrøm $I_{bal}$    & $50\milli\ampere$     & $42\milli\ampere$     &  Integreret begrænset til én af gangen  \\
			Temperaturmåling               &   Ja                   &  Ja                   &    \\
			Strømbeskyttelse op/afladning  &   Ja                   &  Ja                   &    \\
			Brugerflade (shell)                        &   Ja                   &  Ja                   &    \\
			Kapacitetsberegninger                        &   Ja                   &  Ja                   &   Både vha. spænding og strøm\\
			 % &      &     &    \\

			\bottomrule
		\end{tabular}
		%\begin{tablenotes}
		%	\item[a] \textit{Priser uden fragt.}
		%\end{tablenotes}
		\caption{Sammenligning af specifikationer.}
		\label{tab:specifikationer}
	\end{threeparttable}
\end{table} 
\FloatBlock

Nogle af specifikationerne afviger fra hinanden, og ved balanceringen var det ikke muligt at ramme helt den samme balanceringsstrøm pga. spændingsfaldet over balanceringsmodstanden var forskellig. 

\section{Kostpris} \label{sec:sammenligning_kostpris}

Tabel \ref{tab:pris_diskret} og tabel \ref{tab:pris_ic} viser et oversigt over priserne ved køb af en af hver. Den reelle pris vil være væsentlige mindre med større volumen. Priserne er uden fragt og ved modstande og kondensatorer er en gennemsnitspris anvendt. Prisen er også uden PCB produktion da den varierer rigtig meget afhængig af tid, kvalitet og område det bliver bestilt fra. Alle beløb er taget fra sider og butikker som er tilgængelige for alle og ikke kun virksomheder, og ved køb af kun af gangen. I praksis vil priserne være meget mindre grundet større volumen.

\begin{table}[h!]
	\small
	\centering
	\begin{threeparttable}
		\begin{tabular}{ l l l l l l l }
			\toprule
			\multicolumn{1}{l}{\textbf{Antal}}          &
			\multicolumn{1}{l}{\textbf{Type}}           &
			\multicolumn{1}{l}{\textbf{Pris pr. stk.}}  &
			\multicolumn{1}{l}{\textbf{Pris total}}   \\ 
			\hline
			70 &  Modstande                 &  0.10 DKK   &  7.00 DKK  \\
			15 &  Kondensatorer             &  0.40 DKK   &  6.00 DKK  \\
		    8  &  BC846B                    &  0.30 DKK   &  2.40 DKK  \\
		    6  &  BC856B                    &  0.70 DKK   &  4.20 DKK  \\
		    2  &  LM324-SMD-5GATES          &  2.47 DKK   &  4.94 DKK  \\ 
			2  &  Elektrolyt kondensatorer  &  1.50 DKK   &  3.00 DKK  \\
			2  &  LED-G-3MM                 &  1.03 DKK   &  2.06 DKK  \\
	 	    2  &  1N5819HW                  &  0.43 DKK   &  0.86 DKK  \\
	 	    2  &  SDTM-620-5MM              &  0.30 DKK   &  0.60 DKK  \\ 
		    1  &  STL15DN4F5                & 23.27 DKK   & 23.27 DKK  \\
		    1  &  LM75BIM3                  & 16.28 DKK   & 16.28 DKK  \\
		   	1  &  LPC804M101JHI33           & 10.05 DKK   & 10.05 DKK  \\
		   	1  &  Shunt modstand            &  4.68 DKK   &  4.68 DKK  \\
		   	1  &  TS9011SCX                 &  0.90 DKK   &  0.90 DKK  \\
  			1  &  2N7002                    &  0.30 DKK   &  0.30 DKK  \\
	  	    1  &  BZX84-C12                 &  0.20 DKK   &  0.20 DKK  \\
	        \hline
	  	   116 &                            & Total       & 86.74 DKK  \\  
			\hline
			\bottomrule
		\end{tabular}
		\begin{tablenotes}
		\item[a] \textit{Priser uden fragt.}
		\end{tablenotes}
		\caption{Kostpris på komponenter til den diskrete version.}
		\label{tab:pris_diskret}
	\end{threeparttable}
\end{table} 
\FloatBlock



\begin{table}[h!]
	\small
	\centering
	\begin{threeparttable}
		\begin{tabular}{ l l l l l l l }
			\toprule
			\multicolumn{1}{l}{\textbf{Antal}}        &
			\multicolumn{1}{l}{\textbf{Type}}         &
			\multicolumn{1}{l}{\textbf{Pris pr stk}}  &
			\multicolumn{1}{l}{\textbf{Pris total}}   \\ 
			\hline
			35 &  Modstande                 &  0.10 DKK   &  3.50 DKK  \\
			21 &  Kondensatorer             &  0.40 DKK   &  8.40 DKK  \\
			4  &  Elektrolyt kondensatorer  &  1.50 DKK   &  6.00 DKK  \\
			3  &  SDTM-620-5MM              &  0.30 DKK   &  0.90 DKK  \\
			2  &  LED-G-3MM                 &  1.03 DKK   &  2.06 DKK  \\
			1  &  STL15DN4F5                & 23.27 DKK   & 23.27 DKK  \\
			1  &  BQ7692003PWR              & 18.28 DKK   & 18.28 DKK  \\
			1  &  LM75BIM3                  & 16.28 DKK   & 16.28 DKK  \\
			1  &  LPC804M101JDH20           &  9.81 DKK   &  9.81 DKK  \\
			1  &  Shunt modstand            &  4.68 DKK   &  4.68 DKK  \\
			1  &  TS9011SCX                 &  0.90 DKK   &  0.90 DKK  \\
			1  &  BSS84                     &  0.80 DKK   &  0.80 DKK  \\
			2  &  BC846B                    &  0.60 DKK   &  0.60 DKK  \\
			1  &  1N5819HW                &  0.43 DKK   &  0.43 DKK  \\
			1  &  2N7002                    &  0.30 DKK   &  0.30 DKK  \\
			1  &  BZX84-C12                 &  0.20 DKK   &  0.20 DKK  \\  
			\hline
		   77  &                            & Total       & 96.41 DKK  \\  
			\hline
			\bottomrule
		\end{tabular}
		\begin{tablenotes}
			\item[a] \textit{Priser uden fragt.}
		\end{tablenotes}
		\caption{Kostpris på komponenter til versionen med batteriovervågningskreds.}
		\label{tab:pris_ic}
	\end{threeparttable}
\end{table} 
\FloatBlock

\section{Delkonklusion}
Der kan ud fra tabel \ref{tab:specifikationer} ses, at de to versioner kan det samme. Dog er den integrerede version begrænset på modularitet, da den ikke kan balancere celler der er placeret ved siden af hinanden\footnote{Celler der \textbf{ikke} er placeret ved siden af hinanden kan balanceres på samme tid ifølge databladet side 30: \url{http://www.ti.com/lit/ds/symlink/bq76920.pdf}}. Dertil er dens maksimale balanceringsstrøm heller ikke særlig stor medmindre ekstern balancering udvikles. \\

Det maksimale antal tilkoblede celler er fem, medmindre en større version af kredsen vælges. På den diskrete skal der tilføjes et eller flere "sæt"\space af balanceringskredsløb hvis flere celler ønskes. Det vil sige at begge versioner videreudvikles, men den integrerede kræver noget ombygning da en anden (større) kreds vil være nødvendig. På den diskrete er platformen den samme, da den samme elektronik kan tilføjes.\\

På trods af den integrerede versions begrænsninger kræver den mindre plads, da den klarer mange opgaver internt som den anden version har ekstra komponenter til. Dette kan være med til at sænke prisen på PCB'en på den integrerede. Det giver også mindre komponenter alt i alt, men som der ses ender det ud i at den diskrete alligevel er billigere, da transistorkøb er billigere end at købe en hel kreds ekstra.\\