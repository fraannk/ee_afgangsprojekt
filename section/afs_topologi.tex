\section{Om batteristyresystemer}
Et batteristyresystem er et elektronisk system, der styrer de enkelte battericeller eller hele batteripakken. Den overvåger og holder styr på cellens tilstand, herunder spænding, kapacitet, ladning og temperatur. Dertil søger systemet for at holde cellerne beskyttet i et sikkert operation område (Safe Operating Area). Systemet sørger for at batteripakkens celler er i balance med hinanden, hvilket praktisk betyder at cellespændingerne er tilnærmelsesvis ens.
\\

\section{Centraliseret system}

\section{Decentraliseret system}

\section{Modulær system}

\section{Valg af topologi}

\subsection{Vægtningsskema}

\begin{itemize}
	\item Virkningsgrad
	\item Sikkerhed
	\item Modularitet
	\item Brugertilpasning
\end{itemize}