\section{Om batteristyresystemer}
Et batteristyresystem er et elektronisk system, der styrer de enkelte battericeller eller hele batteripakken. Den overvåger og holder styr på cellens tilstand herunder spænding, kapacitet, ladning og temperatur. Dertil søger systemet for at holde cellerne beskyttet i et sikkert operation område (Safe Operating Area). Systemet sørger for at batteripakkens celler er i balance med hinanden, hvilket praktisk betyder at cellespændingerne er tilnærmelsesvis ens. I tilfælde af at parametre skulle overstige deres max værdi, skal systemet søge for at afbryde batteriet fra belastningen eller opladeren.
\\

De primære formål med en BMS er derfor som følgende:
\begin{itemize}[noitemsep]
	\item Den minimere risikoen for beskadigelse af cellerne
	\item Den skal overvåge og kontrollere batteripakkens op- og afladningsprocess
\end{itemize}


\section{Funktionaliteter i et batteristyresystem}
Herunder vil de mest centrale funktionaliteter i projektet blive beskrevet.

\subsection{Op- og afladningskontrol}
Battericeller bliver oftest skadet ved forkert op- og afladning. Derfor er netop disse to punkter en essentiel opgave for et batteristyresystem.

\subsection{Cellebalancering}
Ideelt set er alle battericeller produceret ens med samme egenskaber og opfører sig ens under hele cellens leve tid. Desværre er det langt fra virkeligheden, hvor der er produktionstolerancer, hvor nogle celler vil være stærkere end andre. I praksis betyder det forskelle på minimal, nominal og maksimal spænding samt forskelle i kapacitet. I batteripakker hvor flere celler indgår, vil forskellen på stærke og svage celler blive forstørret med hver lade-oplade cyklus. Under opladning kan en svag celle blive svagere indtil den fejler, hvilket kan forsage permanent skade på hele batteripakken.
\\
Ved hjælp af cellebalancering kan forskelle på cellerne mindskes og derved forøge en batteripakkes levetid.

\subsubsection{Passiv balancering}

\subsubsection{Aktiv balancering}

\subsection{State of Charge (SoC)}
\klp{Mangler at få styr på denne sektion}
En af de vigtigste funktionaliteter i et batteristyresystem er estimering af State of Charge, og er direkte defineret som den procentdel af maksimal ladning i en celle eller en batteripakke. SoC er vigtig da det afspejler batteripakkens ydeevne. Nøjagtig estimering af SoC er ikke kun vigtig til beskyttelse af cellerne, men også for at forhindre overopladning og underafladning, hvilket bidrager til forlængelse af batteripakkens levetid. Dertil muliggør informationen også at lade batteristyresystemet ligge rationelle strategier for at spare energi.
\\
Informationen kan også bruges til at indikere behov tilslutning af oplader til brugeren gennem f.eks. en statusindikator eller om afladning af systemet kan tillades. 
Ved strømsvigt hvor f.eks et UPS-anlæg (Uninterruptible power supply) skal levere nødstrøm indtil normal drift genoprettes, vil SoC bidrage med information om hvorvidt anlægget overhovedet vil være i stand til at understøtte den tilkoblede belastning.
\\

SoC i en celle defineres som forholdet mellem nuværende ladning $Q_{t}$ og $Q_{n}$, som er den nominelle ladning og er givet fra producenten, og er den maksimale ladning der kan opbevares i en celle.

\begin {equation} 
SoC(t) = \frac{Qt}{Q_n} \label{eq:soc}
\end {equation}

Der anvendes typisk tre metoder til at afgøre SoC, gennem en direkte måling af cellespænding, ved at tælle ladning, og en blanding af disse to.

\subsubsection{Direkte måling}
Ved denne metode vil en simpel spændingsmåling henover hver cellerne kunne indikere kapaciteten, da spændingen en celle tilnærmelsesvis falder lineært under afladning i dette projekts valgte arbejdsområde.

\subsubsection{Coulomb-tælle metode}
Denne metode går ud på at måle afladningsstrømmen i en batteripakke og integrere strømmen over tid, for at estimere realtids-kapaciteten $SoC(t)$. 

\begin {equation} 
SoC(t) = SoC(t-1)+\frac{I(t)}{Q_n} \Delta t  \label{eq:coulomb-count}
\end {equation}

\begin {equation} 
SoC(t) = SoC(0) - \frac{1}{C} \int_{0}^{t} idt  \label{eq:coulomb-count_1}
\end {equation}


\subsubsection{Kombination}


\subsection{State of Health (SoH)}
Formålet med SoH er at give en indikation af hvilken ydeevne der kan forventes af en celler eller batteripakke i den nuværende tilstand, eller indikere hvor den resterende levetid før der er behov for udskiftning.
\\
I nogle applikationer bliver SoH løbende gemt i en logbog for validere

\subsection{Temperaturmåling}
Måling af temperature i en batteripakke er nødvendig, for at sikre optimal op- og afladning. 

\subsection{Kortslutningsbeskyttelse}
I tilfælde af fejl ved op- eller afladning, skal BMS'en kunne afbryde forbindelsen mellem opladeren eller belastningen. Det sker typisk ved hjælp af en cuttoff-FET. Opførelsen af cutoff-FET'en er typisk defineret ud fra parametre sat ud fra max værdier af cellespændinger, strøm målinger og realtids detektering. 

Batteristyresystemer kan opdeles i tre topologier, og bliver gennemgået herunder.
\section{Centraliseret system}

\section{Decentraliseret system}

\section{Modulær system}

\section{Valg af topologi}

\subsection{Vægtningsskema}

\begin{itemize}
	\item Virkningsgrad
	\item Sikkerhed
	\item Modularitet
	\item Brugertilpasning
\end{itemize}