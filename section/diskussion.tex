\chapter{Diskussion og vurdering}\label{kap:diskussion}
Den valgte topologi blev den modulære BMS, da det var den ideelle løsning til antallet af celler i dette system. Da mange funktionaliteter skulle implementeres blev passiv balancering valgt, grundet dens simplicitet, men for at optimere på systemet i fremtiden, kunne der vælges en mere effektiv topologi såsom "controlled shunting". \\ 

Strømmålingen blev realiseret i begge versioner med en $10\milli\ohm$ shunt modstand. I versionen med batteriovervågningskredsen anbefales dog en $1\milli\ohm$ shunt modstand, hvis der skal trækkes meget store strømme, hvilket ikke blev nødvendigt i projeket. Nøjagtigheden til estimering af batteripakkernes kapacitet kunne være blevet bedre, såfremt andre teknikker blev taget i spil. Da kun detektering af batteripakkens kapacitet i yderpunkterne var nødvendig, blev der valgt at fokusere på videre udvikling af resten af funktionaliteterne. \\

Omregning af strømmålingen i software viste sig at have en fejl under opladning. En formodning om fejlen er at konverteringen fra 16-bit 2's complement tal til standard tal går galt. Konverteringen finder sted for at kunne sende til host, samt at bruge variablerne i senere udregninger. Det tyder på at fejlen ligger i sign-bit delen, men pga. tidspres nåede dette ikke at blive realiseret korrekt. \\

Da PCB'erne ankom viste det sig, at et forkert footprint til microcontrolleren var valgt på den diskrete version, hvilket gjorde at den ikke kunne monteres. Test af software til den diskrete version kunne derfor ikke ske. Da de to versioner anvender samme microcontroller serie, blev der vurderet at hvis funktionaliteterne og "hovedrutinen"\space virkede på den integrerede, ville de også virke på den diskrete (med ændringer og tilpasning). Da temperatursensoren virker på samme måde i begge versioner, blev der også vurderet at denne også ville virke på den diskrete. Balancering, samt opladning og afladning blev testet på begge versioner, dog manuelt på den diskrete pga. dette. \\

For at opnå et lavere strømforbrug når batteripakken ikke er under drift, kan der undersøges om standby og/eller deep-sleep modes på microcontrolleren. Andre hardwaremæssige tiltag kan også realiseres, som f.eks. færre antal opamps til måling af cellespænding eller deaktivering når der ikke er behov for at læse cellespændingen. \\
