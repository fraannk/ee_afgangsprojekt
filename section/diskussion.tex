\chapter{Diskussion og vurdering}\label{kap:diskussion}
Valg af topologi blev modulær, da det var den ideelle løsning til antallet af celle i systemet. Da mange funktionaliteter skulle implementeres blev passiv balancering valgt, grundet dens simplicitet og effektivitet for balancering af cellerne.
\\

Strømmålingen blev realiseret i begge versioner med en $10\milli\ohm$ shunt modstand. I versionen med battery monitor kredsen anbefales en $1\milli\ohm$ shunt modstand, da den skal kunne måle en meget større strøm end nødvendig i dette projekt. Nøjagtigheden til estimering af batteripakkernes kapacitet kunne være blevet bedre, såfremt andre teknikker blev taget i spil. Da kun detektering af batteripakkens kapacitet i yderpunkterne var nødvendig, blev der valgt at fokusere på videre udvikling af resten af funktionaliteterne.
\\

Der blev fundet fejl i schematics på den diskret opbyggede version, da PCB var bestilt. Sikkerhedskredsen til afladning var forbundet med oplade MOSFET'en og omvendt. Dertil blev der valgt forkert footprint til microcontrolleren på den diskrete version, hvilket gjorde at den ikke kunne monteres. Test af software til den diskrete version kunne derfor ikke ske. Da de to versioner anvender samme microcontroller serie, blev der vurderet at en fejlfinding af softwaren kunne føre til en funktionsdygtig microcontroller. Da temperatursensoren virker på samme måde i begge versioner, blev der også vurderet at denne også ville virke på den diskrete. Balancering, samt opladning og afladning blev testet på begge versioner, hvortil begge nåede de opsatte krav.
