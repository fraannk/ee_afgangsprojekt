\chapter*{Forord}\label{chap:forord}
\addcontentsline{toc}{chapter}{Forord}

Udarbejdningen af dette afgangsprojekt er foretaget af studerende på 7. semester elektronik og datateknik, ved Syddansk Universitet, Teknisk Fakultet, efteråret 2018. Projektet er en sammenfatning af nogle af de opnåede fagligheder fra uddannelsens undervisning. Afleveringen af rapporten er dertil et krav for at kunne gå til eksamen.\\

Som udgangspunkt antages det at læseren har et tilsvarende fagligt niveau som en 7. semester studerende med linjefag indenfor elektronik og datateknik.
Rapporten tager derfor ikke hensyn til at skulle forklare den grundliggende viden, en studerende med samme faglige niveau antages at have.

\subsection{Læsevejledning}
Denne rapport er opbygget således, at den læses fra start til slut. Rapportens struktur er opdelt i hovedsektioner og undersektioner. Opbygningen er taksonomisk både i kapitelstrukturer og som helhed.

\subsection{Arbejdsfordeling}
Arbejdsfordelingen har hovedsageligt været delt op i hardware og software. Kenneth har taget sig af hardwaren, mens Søren har udvilket softwaren. Der har gennem hele forløbet været tæt samarbejde, for at opnå en fuld forståelse af hele systemet for begge parter. 

\subsection{Typografiske konventioner}
Her er en kort oversigt over de typografiske konventioner der anvende i denne rapport\\
\begin{tabular}{l p{0.6\linewidth}}
	\textit{Kursiv tekst}			& Angiver filnavne i den tilhørende kodebase samt fremhævelse af ord eller fagudtryk. \\
	\textbf{Fed tekst}				& Bruges til a fremhæve produkt eller system specifikke betegnelser.\\
	\texttt{Konstant brede tekst}	& Anvendes til kildekode eksempler. Ligeledes anvendes afgrænsende områder.\\
	\emph{Fremhævet tekst}		    & Bliver brugt når der gives en kort introduktion til hvert kapitel.\\
\end{tabular}

\subsection{Typografi}
Rapporten har en opbygning, der gør den behagelig at læse og indeholder derfor mange tomme sider der er angivet med sidetal.
