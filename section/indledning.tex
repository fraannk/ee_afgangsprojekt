\chapter{Indledning}

\emph{Lithium batterier er højtydende, langtidsholdbare og fås i forskellige former, hvilket gør dem velegnet til mange forskellige applikationer. De har dog ulempen at hvis de ikke overvåges kan de hurtigt forgå eller, i værste tilfælde, kan der opstå farlige situationer. Derfor bruges der et batteristyresystem til opladning, afladning og vedligeholdelse. Batteristyresystemet overvåger konstant batteriet, for at sikre sig at alle dets parametre er i orden. Er der flere batteri-celler i serie, holder systemet dem også i balance. }

\section{Forord}\label{sec:forord}
forord

\section{Formål}
Formålet med dette afgangsprojekt er at udarbejde et produkt der efterviser nogle af de opnåede fagligheder igennem uddannelsen som Diplomingeniør i Elektronik og Datateknik.

\section{Problemformulering}


\section{Kravspecifikation} \label{afs:kravspecifikation}

\subsection{Funktionaliteter}

\begin{itemize}
\item Opladning og afladning af Lithium celler
\item Balancering under opladning og afladning v/ flere celler i serie
\item Afladningsbeskyttelse/kortslutningsbeskyttelse
\item Maksimal lade- og afladestrøm
\item Temperaturmålinger
\item State of charge
\item State of health
\end{itemize}

\subsection{Vægtningsskema}

\section{Projektafgrænsning}
Bliver udafbejdet senenere

\section{Løsningsmodel}
Bliver udafbejdet senenere

\section{Læsevejledning}
Bliver udafbejdet senenere

\section{Proces- og arbejdsmetode}
Bliver udafbejdet senenere
