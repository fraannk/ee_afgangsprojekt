\chapter{Indledning}

\emph{Lithium batterier er højtydende, langtidsholdbare og fås i forskellige former, hvilket gør dem velegnet til mange forskellige applikationer. De har dog ulempen at hvis de ikke overvåges kan de hurtigt forgå eller, i værste fald, kan der opstå farlige situationer. Derfor kan der med fordel anvendes et batteristyresystem til opladning, afladning og vedligeholdelse. Batteristyresystemet overvåger konstant de individuelle celler og batteriet som helhed, for at sikre at parametrene er i orden. I denne rapport vil læseren blive guidet igennem processen bag udviklingen af et batteristyresystem.}

\section{Formål}
Formålet med dette afgangsprojekt er at udarbejde et produkt, der efterviser nogle af de opnåede fagligheder igennem uddannelsen som Diplomingeniør i Elektronik og Datateknik - både teoretisk og praktisk.

\section{Problemformulering}
Linak er gået ind på markedet indenfor reclinerstole, hvor et nyt aktuatorsystem er under udvikling. Da en batteripakke vil indgå i det samlede system, og den består af en seriekobling af lithium celler, er der brug for et batteristyresystem. Batteristyresystemet (BMS’en) vil sidde mellem batterierne og elektronikken for at styre op- og afladning af batteripakken, samt bidrage med sikkerhed i batteripakken. Problemformuleringen er derfor som følgende:\\

Et batteristyresystem skal udvikles og realiseres ved hjælp af en microcontroller og integrerede kredse eller transistorer. Det ønskes at være muligt at se aktuel batteristatus via et display eller en konsol samt at batteristyresystemet kan overvåge 4 celler i serie. \\

Der ønskes også svar på følgende problemstillinger: 
\begin{itemize}[noitemsep]
	\item Hvordan kan en BMS opbygges ved hjælp af integrerede kredse?
	\item Hvordan kan en BMS opbygges diskret? 
	\item Hvordan fremstilles software til kontrol af batteripakken?
	\item Hvilken topologi af batteristyresystemer egner sig bedst til løsning af Linak's problemstilling? 
	\item Hvordan sikres der imod kortslutning og overcurrent/overvoltage? 
\end{itemize}

\section{Projektafgrænsning}
Et batteristyresystem kan realiseres på forskellige måder. I dette projekt vil forskellige topologier blive undersøgt, hvorefter den bedst egnede ud fra vægtningsskemaet, som ses i tabel XXXXXX, vil blive udviklet og realiseret. \\

Projektet går ud på, at realisere to prototyper af samme topologi. Den ene prototype bliver bygget op omkring en færdig balanceringskreds, hvor den anden prototype vil realiseres diskret. De to prototyper vil efterfølgende blive analyseret og sammenlignet ud fra skemaet, som ses i tabel XXXXX, for 


Under dette projekt designes et batteristyresystem, som skal kunne balancere en seriekobling af fire 18650 Lithium-Ion celler. Der tilstræbes at designe og konstruere en funktionel prototype.

\subsection{Kravspecifikation} \label{afs:kravspecifikation}
I dette afsnit fremstår de opsatte krav til projektet. Kravene er en sammenfatning af ..... \\

\begin{itemize}[noitemsep]
	\item Max afladestrøm, $I_{out} = 8\ampere$
	\item Max opladestrøm, $I_{in} = 1*C$ (kapacitet)
	\item Afladning af batteripakken under opladning
	\item Balancering under opladning og afladning flere celler i serie. Balanceringsspændingen skal have en opløsningen på  $V_{bal} = 50\milli\volt$
	\item "Låsning" af yderpunkter på batteriernes afladekurve (Brug af 80\%)
	\item Afladnings- og kortslutningsbeskyttelse
	\item Temperaturmålinger af celler for sikker drift $\SI{-20}{\celsius}$ til $\SI{50}{\celsius}$
	\item State of charge (SOC) - Cellernes kapacitet
	\item State of health (SOH) - Antal fulde lade cycles
	\item Softwaren er en del af produktet og skal overholde ANSI(C90) standarden
\end{itemize}

\subsection{Vægtningsskema}

\section{Løsningsmodel}
Bliver udarbejdet senenere

\section{Læsevejledning}
Bliver udarbejdet senenere

\section{Proces- og arbejdsmetode}
Projektarbejdet er delt op i tre faser. Research og forberedelse, udvikling og test, færdiggørelse og konklusion. I research-fasen samles alt relevant information omkring emnet for at opnå en bedre forståelse. I udviklingsfasen fokuseres der på følgende punkter: 
\begin{itemize}
	\item Virkningsgrad
	\item Sikkerhed
	\item Modularitet
	\item Brugertilpasning
\end{itemize}


