\chapter{Konklusion} \label{kap:konklusion}
Der kan igennem arbejdet med dette afgangsprojekt konkluderes, at til et system med få antal celler, egner den modulære topologi, og dens enkelthed, sig bedst. \\

Et batteristyresystem kan opbygges ved brug af en microcontroller og batteriovervågningskredsen, BQ76920. Disse kan kommunikere ved hjælp af $I^2C$ og overvåger konstant cellespændinger og strømforbrug. Overvågningskredsen står også for balanceringen, men balanceringsalgoritmen styres af microcontrolleren. \\

Et batteristyresystem kan dertil også opbygges uden brug af den førnævnte batteriovervågningskreds ved hjælp af transistorkoblinger og operationsforstærkere til håndtering af logikken i samspil med en microcontroller. Microcontrolleren kan her også styre balanceringen og opladning/afladning, og hardwaren vil reagere hurtigere end hvad microcontrolleren kan, hvilket fungerer som et ekstra sikkerhedslag. En temperatursensor kan tilføjes på begge versioner af BMS'en for overvågning af pakkens temperatur og derved kontrollere hvornår der må oplades og aflades. \\

Softwaren kan fremstilles med udgangspunkt i driver-biblioteket tilhørende NXP LPC804 serien. For at kommunikation med host og eventuel batteriovervågningskreds kan finde sted, kan diverse kommunikationsprotokoller sættes op. Evaluationboards kan fremskynde udviklingen af software før hardware er tilgængeligt. 